\section{Einleitung}

\begin{frame}
  \frametitle{Definition: Primfaktorzerlegung}
  \begin{definition}
    Die Darstellung
    \begin{equation*}
      N=\prod\limits_{i=1}^M p_i^{m_i}.
    \end{equation*}
    einer Zahl $N\in\mathbb{N}$ mit Primzahlen $p_1,\dots,p_M$ mit $p_i \neq p_j$ für $i \neq j$ und Exponenten $m_1,\dots,m_M\in\mathbb{N}$ ist die \textit{Primfaktorzerlegung} der Zahl $N$.
  \end{definition}
  \pause{}

  Zerlegung kann rekursiv aufgebaut werden:
  \begin{equation*}
    N=A\cdot B
  \end{equation*}
  \Rightarrow{} $A, B$ weiterzerlegen
\end{frame}

\begin{frame}
  \frametitle{Bedeutung der Primfaktorzerlegung}
  \begin{itemize}
    \item Primfaktorzerlegung wichtig in der Kryptografie
    \begin{itemize}
      \item Test
    \end{itemize}
    \item ``Harte Probleme'' generell interessant
  \end{itemize}
\end{frame}

\begin{frame}
  \frametitle{Lösungsansätze}
  \begin{itemize}
    \item ``Brute Force'' (durchprobieren aller Lösungen)
    \item Zahlkörpersieb $\mathcal{O}\left(\exp\left(c\cdot {\left(\log n\right)}^{\frac{2}{3}}{\left(\log\log n\right)}^{\frac{1}{3}}\right)\right)$
    \item Shor-Algorithmus auf Quantencomputern $\mathcal{O}\left(n^2 \log n \log\log n\right)$\cite{shor}
    \item Adiabatic Quantum Computing
    \item Simulated Annealing
  \end{itemize}
\end{frame}
