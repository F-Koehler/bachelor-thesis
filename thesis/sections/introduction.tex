\chapter{Einleitung}
Die Zerlegung einer Zahl in ihre Primfaktoren lässt sich auf einem klassischen Computer mit keinem bisher bekanntem Algorithmus effizient, d.h.\ in polynomieller Laufzeit, berechnen. Das schnellste klassische Verfahren, das Zahlkörpersieb hat eine exponentielle Laufzeit~\parencite{pomerance}.\\
Die Probe, ob mehrere Zahlen die Primfaktoren einer Zahl sind, hingegen lässt sich effektiv durch einfache Multiplikation durchführen. Auf diesem Prinzip basiert eine Vielzahl aktueller Verschlüsselungsmethoden, z.B.\ das RSA-Kryptosystem~\parencite{rsa}. \\
Jenseits der klassischen Informatik gibt es jedoch schon Verfahren, die eine effektive Behandlung des Problemes verheißen. So schlug P. W. Shor 1994 einen Algorithmus~\parencite{shor} vor, der auf einem Quantencomputer in polynomieller Laufzeit Primfaktoren einer Zahl berechnet. Die Implementierung eines entsprechenden Quantencomputer ist schwierig und bisher nur für relativ kleine Systeme realisiert worden, so dass mit diesem Verfahren erst kleine Zahlen wie $15$ zerlegt wurden~\parencite{vandersypen}.\\
Ein weiterer Ansatz nutzt Adiabatic Quantum Computing~\parencite{suter} um sich dem Problem zu nähern. Mit diesem Verfahren konnte die Zahl $143$ faktorisiert werden~\parencite{xu}. \\
Hier soll nun ein von E. L. Altschuler und T. J. Williams entwickeltes Verfahren~\parencite{altschuler} breschrieben und untersucht werden.\\
Zunächst wird in Kapitel~\ref{ch:algorithm} ein knapper Überblick über die Methode des Simulated Annealings gegeben und anschließend die Anwendung dieser Methode auf das Problem der Primfaktorzerlegung beschrieben. Desweiteren wird die Laufzeit des Algortihmus abgeschätzt.\\
In Kapitel~\ref{ch:results} wird die Implementierung der Programme beschrieben. Außerdem werden die durchgeführten Untersuchungen beschrieben und ausgewertet.\\
In Kapitel~\ref{ch:conclusion} schließlich werden die Resultate noch einmal zusammengefasst und weitere Untersuchungen beschrieben, die aus Zeitgründen nicht mehr durchgeführt werden konnten.
