\chapter{Einleitung}
\setcounter{page}{1}
Die Zerlegung einer Zahl in ihre Primfaktoren lässt sich auf einem klassischen Computer mit keinem bisher bekanntem Algorithmus effizient, d.h.\ in polynomieller Laufzeit, berechnen. Das schnellste klassische Verfahren, das Zahlkörpersieb, hat eine exponentielle Laufzeit~\parencite{pomerance}.\\
Die Probe hingegen, ob mehrere Zahlen die Primfaktoren einer Zahl sind, lässt sich effektiv durch einfache Multiplikation durchführen. Auf diesem Prinzip basiert eine Vielzahl aktueller Verschlüsselungsmethoden, z.B.\ das RSA-Kryptosystem~\parencite{rsa}. \\
Jenseits der klassischen Informatik gibt es jedoch schon Verfahren, die eine effektive Behandlung des Problemes verheißen. So schlug P. W. Shor 1994 einen Algorithmus vor~\parencite{shor}, der auf einem Quantencomputer in polynomieller Laufzeit Primfaktoren einer Zahl berechnet. Die Implementierung eines entsprechenden Quantencomputers ist schwierig und bisher nur für relativ kleine Systeme realisiert worden. Die prinzipielle Funktion dieses Algorithmus wurde für die Zahl $15$ experimentell bestätigt worden~\parencite{vandersypen}.\\
Ein weiterer Ansatz nutzt Adiabatic Quantum Computing~\parencite{suter} in einem NMR-System. Mit diesem Verfahren konnte die Zahl $143$ faktorisiert werden~\parencite{xu}. \\
Hier soll nun ein von E. L. Altschuler und T. J. Williams entwickeltes Verfahren~\parencite{altschuler} beschrieben und untersucht werden, das mit Simulated Annealing eine aus der statistischen Physik motivierte und in vielen Bereichen genutzte Optimierungsmethode verwendet. Dieses Verfahren ist also nicht deterministisch und führt nicht immer zum gesuchten Ergebnis. Dies gilt allerdings auch für den Shor-Algorithmus.\\
Zunächst wird in Kapitel~\ref{ch:algorithm} ein knapper Überblick über die Methode des Simulated Annealings gegeben und anschließend die Anwendung dieser Methode auf das Problem der Primfaktorzerlegung beschrieben. Desweiteren wird die Laufzeit des Algortihmus abgeschätzt und mit der des Zahlkörpersiebs verglichen.\\
In Kapitel~\ref{ch:results} wird die Implementierung der Programme erläutert. Außerdem werden die durchgeführten Untersuchungen beschrieben und ausgewertet.\\
In Kapitel~\ref{ch:conclusion} werden die Resultate noch einmal zusammengefasst und weitere Untersuchungen beschrieben, die nicht mehr durchgeführt wurden.
