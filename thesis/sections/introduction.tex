\chapter{Einleitung}
Die Zerlegung einer Zahl in ihre Primfaktoren lässt sich auf einem klassischen Computer mit keinem bisher bekanntem Algorithmus effizient, d.h.\ in polynomieller Laufzeit, berechnen. Der schnellste klassische Algorithmus, das Zahlkörpersieb hat eine Laufzeit von $\approx\exp\left(c\cdot{\left(\log n\right)}^\frac{2}{3}{\left(\log\log n\right)}^\frac{1}{3}\right)$~\parencite{pomerance}.\\
Die Probe, ob mehrere Zahlen die Primfaktoren einer Zahl sind, hingegen lässt sich effektiv durch einfache Multiplikation durchführen. Auf diesem Prinzip basieren eine Vielzahl aktueller Verschlüsselungsmethoden, z.B.\ das RSA-Kryptosystem~\parencite{rsa}. \\
Jenseits der klassischen Informatik gibt es jedoch schon Verfahren, die eine effektive Behandlung des Problemes verheißen. So schlug P. W. Shor 1994 einen Algorithmus~\parencite{shor} vor, der auf einem Quantencomputer innerhalb von $\mathcal{O}\left({\left(\log n\right)}^3\right)$ einen Primfaktor einer Zahl berechnet. Dieser Algorithmus benötigt jedoch $\log n$ Qubits zur Berechnung, so dass mit diesem Verfahren erst kleine Zahlen wie $15$ zerlegt wurden~\parencite{vandersypen}.\\
Ein weiterer Ansatz nutzt Adiabatic Quantum Computing~\parencite{suter} um sich dem Problem zu nähern. Mit diesem Verfahren konnte die Zahl $143$ faktorisiert werden~\parencite{xu}. \\
Hier soll nun ein von E. L. Altschuler und T. J. Williams entwickeltes Verfahren~\parencite{altschuler} breschrieben und untersucht werden.
