\begin{abstract}
  \vfill

  \selectlanguage{english}
		{\large\textbf{Abstract}}\\
		Prime factorization is an interesting problem as it is not efficiently solvable on a classical computer with any known algorithm. This fact is used in modern cryptograpy methods. In this thesis, a method based on simulated annealing will examined. It could be shown that one factorization can be done in time polynomial time.

  \vfill

  \selectlanguage{ngerman}
		{\large\textbf{Zusammenfassung}}\\
		Die Primfaktozerlegung ist ein interessantes Problem, weil es auf einem klassischen Computer mit keinem bekannten Algorithmus effizient lösbar ist. Diese Tatsache wird zum Beispiel in modernen Verschlüsselungsverfahren benutzt. In dieser Arbeit wird eine auf Simulated Annealing basierende Methode untersucht~\parencite{altschuler}. Dabei kann unter anderem bestätigt werden, dass damit ein Schritt der Zerlegung einer Zahl in polynomieller Laufzeit möglich ist.

		\vfill
\end{abstract}
