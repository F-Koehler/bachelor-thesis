\chapter{Beschreibung des Verfahrens}\label{ch:algorithm}



\section{Primfaktorzerlegung}
Das hier beschriebene Verfahren zur Faktorisierung basiert größtenteils auf der Arbeit von Altschuler und Williams~\parencite{altschuler}. \\
Die Primfaktorzerlegung bezeichnet die eindeutige Darstellung einer Zahl $N\in\mathbb{N}$ durch
\begin{equation*}
  N=\prod\limits_{i=1}^M p_i^{m_i},
\end{equation*}
wobei die $p_i\in\mathbb{N}$ mit $p_i>1$ und $p_i\neq p_j$ für $i\neq j$ Primzahlen und die $m_i\in\mathbb{N}$ die zugehörigen Exponenten sind.\\
Diese Zerlegung kann rekursiv aufgebaut werden. Zunächst wird die Zahl $N$ in zwei Faktoren $A$ und $B$ zerlegt. Danach werden $A$ und $B$ zerlegt sofern sie nicht prim sind.\\
Es soll ein Verfahren entwickelt werden, welches die Zerlegung $N=A\cdot B$ mit $A\geq B$ berechnet und dabei die Methode des Simulated Annealing anwendet.\\
Im folgenden sollen zunächst einige Eigenschaften der binären Repräsentation der Zahlen $A$ und $B$ abgeleitet werden.



\section{\texorpdfstring{Abschätzung der Wertebereiche von $a,a_1,b,b_1$}{Abschätzung der Wertebereiche von a, a1, b, b1}}\label{sec:parameterspace}
Die Zahlen $N,A,B$ werden im Dualsystem dargestellt. $a,b,n$ sollen dabei angeben, wie viele Stellen dafür unter Vernachlässigung führender Nullen erforderlich sind. $a_1$ bzw.\ $b_1$ geben dabei an, wie häufig die Ziffer $1$ in $A$ bzw. $B$ vorkommt. \\
Der Algorithmus durchläuft alle möglichen Tupel $\left(a,a_1,b,b_1\right)$ (vgl.\ Abschnitt~\ref{sec:annealfac}). Deshalb sollte geprüft werden, inwieweit der Suchbereich bei gegebenem $N$ eingeschränkt werden kann. \\
Die binäre Darstellung von $A$ bzw.\ $B$ kann maximal so lang sein wie die von $N$, d.h.\ $a\leq n\:\wedge\: b\leq n$. Wegen $p_i > 1$ ist $a>1\:\wedge\:b>1$. Es gibt allerdings noch weiteres Potential, die Wertebereiche für $a$ und $b$ zu reduzieren. \\
Durch $a,b,n$ sind die Wertebereiche der Zahlen $A,B,N$ auf
\begin{align*}
		2^{a-1}\leq &A \leq 2^a-1 \\
		2^{b-1}\leq &B \leq 2^b-1 \\
		2^{n-1}\leq &N \leq 2^n-1
\end{align*}
eingeschränkt. Es ist $A\cdot B=N$, sodass bei maximalem $N$
\begin{equation*}
		\big(2^a-1\big)\big(2^b-1\big)=2^{a+b}-2^a-2^b+1<2^{a+b} \\
\end{equation*}
gilt und $a+b$ Bits zur Darstellung von $N$ genügen. Für den Fall eines minimalen $N$ gilt analog
\begin{align*}
		2^{a-1}\cdot 2^{b-1}&=2^{a+b-2}\overset{!}{=}2^{n-1} \\
		\Rightarrow a+b&=n+1
\end{align*}
Bei der Multiplikation $A\cdot B =N$ gilt also $a+b=n\:\vee\:a+b=n+1$. Mit Hilfe dieser Bedingungen kann der Parameterbereich deutlich genauer eingegrenzt werden. \\
Zunächst kann eine untere Schranke für $a$ gewonnen werden, es gilt:
\begin{equation*}
		a=n-b \:\vee\: a=n-b+1
\end{equation*}
Diese Ausdrücke werden minimal für $a=b$, also das maximal zulässige $b \leq a$, sodass der minimale Wert für $a$ durch
\begin{equation}
		a_\mathrm{\min}=\begin{cases}
						2 & \mathrm{falls}\:\lf\frac{n}{2}\rf = 1 \\
						\lf\frac{n}{2}\rf & \mathrm{sonst}
		\end{cases}\label{eq:amin}
\end{equation}
gegeben ist. Dies legt zu gegebenem $a$ die untere Schranke
\begin{equation}
		b_\mathrm{\min}=\begin{cases}
						2 & \mathrm{falls}\:n-a=1 \\
						n-a & \mathrm{sonst}
		\end{cases}\label{eq:amax}
\end{equation}
für $b$ fest.\\
Die binären Darstellungen von $A$ bzw.\ $B$ müssen mindestens eine $1$ enthalten, sonst wäre die Zahl $0$. Maximal können alle Stellen $1$ sein, sodass die Wertebereiche für $a_1$ bzw.\ $b_1$ durch die Ungleichungen
\begin{align*}
		1\leq a_1\leq a \\
		1\leq b_1\leq b
\end{align*}
gegeben sind.\\
Mit diesen Überlegungen wird der Parameterraum von
\begin{align}
		2 \leq a \leq n &\quad\quad 1 \leq a_1 \leq a \notag \\
		2 \leq b \leq a &\quad\quad	1 \leq b_1 \leq b \label{eq:parameterspace1}
\end{align}
auf
\begin{align}
		a_\mathrm{\min} \leq a \leq n &\quad\quad	1 \leq a_1 \leq a \notag \\
		b_\mathrm{\min} \leq b \leq a &\quad\quad	1 \leq b_1 \leq b \label{eq:parameterspace2}
\end{align}
reduziert. Die Auswirkungen auf die Laufzeit werden in Abschnitt~\ref{sec:runtime-theo} betrachtet. \\
Im nächsten Abschnitt wird zunächst ein Überblick über die Methode des Simulated Annealing gegeben.



\section{Simulated Annealing}
Simulated Annealing ist eine Methode, um Optimierungsprobleme zu lösen, die sehr komplex sind~\parencite{nr}. Dies ist der Fall wenn für die Parameter ein großer Wertebereich zulässig ist, die Dimensionalität des Problems, d.h.\ die Anzahl der Parameter, hoch ist oder es neben dem gesuchten globalen Extremum viele lokale Extrema gibt. \\
Zu einer gegebenem Temperatur wird eine adäquate Anfangskonfiguration gewählt, z.B.\ bei einer hohen Temperatur eine zufällige Konfiguration. Dann werden Annealing Schritte ausgeführt, d.h.\ die Temperatur sukzessive gesenkt. Nach jeder Temperatursenkung wird eine ausreichende Anzahl an Metropolisschritten durchgeführt. Dazu wird jeweils eine leichte Modifikation am aktuellen Systemzustand vorgenommen. Verringert die neue Konfiguration die Energie, respektive eine analog dazu definierte Kostenfunktion, so wird die Änderung akzeptiert, ansonsten nur mit einer Wahrscheinlichkeit von $p=\exp\left(-\frac{\Delta E}{k_\mathrm{B}T}\right)$. Auf diese Art und Weise wird versucht, das System zu äquilibrieren.



\section{Anwendung von Simulated Annealing auf die Primfaktorzerlegung}\label{sec:annealfac}
Damit Simulated Annealing auf das das Problem der Primfaktorzerlegung angewendet werden kann, wird zunächst eine Energiedefinition eingeführt. Hier wird
\begin{equation*}
		E\left(A,B,N\right)=\sum\limits_{i=1}^n\begin{cases}
		  		f\left(i\right) & \mathrm{falls}\:{\left\{A\cdot B\right\}}_i={\left\{N\right\}}_i \\
						0 & \mathrm{sonst}\\
				\end{cases}
\end{equation*}
gewählt. Dabei bezeichne die Schreibweise ${\left\{Z\right\}}_i$ die $i$-te Stelle der binären Repräsentation einer Zahl $Z$, wobei $i=0$ das niederwertigste Bit ist. $f\left(i\right)$ ist eine monoton steigende Funktion, z.B. $f\left(i\right)=i$ oder $f\left(i\right)=i^2$. Die Energie wächst mit der Übereinstimmung des Produktes $A\cdot B$ mit $N$, wobei die höherwertigen Bits ein größeres Gewicht haben. Bei dieser Energiedefinition gilt es also ein Maximum zu finden und kein Minimum. \\
Für das Simulated Annealing wird für jedes $4$-Tupel $\left(a,b,a_1,b_1\right)$ mit einer Temperatur $T_0=1$ begonnen, um dann $N_a$ Annealing-Schritte durchzuführen. Dabei wird das System jeweils um einen Faktor $F_c < 1$ abgekühlt, sodass $T_{i+1}=F_c\cdot T_i$. Vor jedem Abkühlen werden $N_c$ Konfigurationen getestet. Dies geschieht mit Hilfe des Metropolis-Algorithmus~\parencite{metropolis}. \\
Zunächst sollen einige notwendige, grundlegende Operationen zur Modifikation von binären Zahlen eingeführt werden. Diese lassen die Länge der Binärzahl und die Anzahl der auf $1$ bzw. $0$ gesetzten Bits invariant und werden verwendet, um Variationen einer Konfiguration zu generieren.
\begin{itemize}
		\item \textbf{swap:} Es werden zwei Bits zufällig vertauscht, wobei darauf zu achten ist, dass eines auf $1$ gesetzt ist und das andere auf $0$.
		\item \textbf{slide:} Es wird eine durchgängige Sequenz an Bits zufällig ausgewählt und das Bit ganz rechts entfernt. Danach werden alle anderen Bits nach rechts durchgeschoben und das entfernte Bit links wieder eingefügt.
		\item \textbf{reverse:} Es wird eine zufällige Sequenz an Bits ausgewählt und ihre Reihenfolge invertiert.
		\item \textbf{random:} Es werden Bits zufällig ausgewählt und zufällig permutiert.
\end{itemize}

\begin{algorithm}[!ht]
		\caption{Metropolis-Schritt}\label{alg:metropolis}
		\begin{algorithmic}[1]
				\Procedure{Metropolis}{$A,B,E,N$}
				\If{$\mathrm{randomInt}\left(0,1\right) = 0$}
  				\State{$A^\prime\gets\mathrm{randomOperation}\left(A\right)$}
				\Else{}
  				\State{$B^\prime\gets\mathrm{randomOperation}\left(B\right)$}
				\EndIf{}
				\If{$A\cdot B=N$}
				  \State{Exit}\Comment{Faktoren $A, B$ wurden gefunden}
				\EndIf{}
				\State{$E^\prime=E\left(A^\prime,B^\prime,N\right)$}
				\If{$E^\prime>E$}
  				\State{$A\gets A^\prime$}
						\State{$B\gets B^\prime$}
						\State{$E\gets E^\prime$}
				\Else{}
  				\State{$p\gets\mathrm{randomFloat}\left(0.0, 1.0\right)$}
						\If{$p<\exp\left(-\frac{E^\prime-E}{k_\mathrm{B} T}\right)$}
    				\State{$A\gets A^\prime$}
		  				\State{$B\gets B^\prime$}
				  		\State{$E\gets E^\prime$}
						\EndIf{}
				\EndIf{}
				\EndProcedure{}
		\end{algorithmic}
\end{algorithm}


Der Ablauf eines Metropolisschrittes ist in Algorithmus~\ref{alg:metropolis} dargestellt. Es wird mit der Funktion $\mathrm{\textit{randomOperation}}\left(\right)$ zufällig eine der Operationen ausgewählt und durchgeführt. Danach wird die Energie $E^\prime$ der daraus resultierenden neuen Konfiguration berechnet. Die Akzeptanzwahrscheinlichkeit für die neue Konfiguration ist
\begin{equation}
		p=\begin{cases}
				1 & \mathrm{falls}\: E^\prime > E \\
				\exp\left(\frac{E^\prime-E}{k_\mathrm{B} T_i}\right) & \mathrm{sonst}
		\end{cases}\label{eq:accept}
\end{equation}
Eine neue Konfiguration wird also auf jeden Fall akzeptiert, wenn sie die Energie vergrößert, ansonsten mit einer über einen Boltzmann-Faktor gegebene Wahrscheinlichkeit. Dabei ist $k_\mathrm{B}$ ein Analogon zur physikalischen Boltzmann-Konstanten. Hier wird sie je nach Problem unterschiedlich gewählt, um das Wachsen der Energieskala mit zunehmendem $n$ auszugleichen, damit eine geeignete Akzeptanzwahrscheinlichkeit ungünstigerer Konfigurationen gewährleistet ist. Dies wird im Abschnitt~\ref{sec:kbguess} genauer untersucht. \\

\begin{algorithm}[ht]
		\caption{Ablauf des Annealing}\label{alg:anneal}
  \begin{algorithmic}[1]
				\Procedure{Anneal}{$A,B,E,N$}
				\State{$T\gets 1.0$}
				\For{$i\gets 1$ to $N_a$}
				  \For{$j\gets 1$ to $N_c$}
						\State{$\mathrm{Metropolis}\left(A,B,E,N\right)$~\ref{alg:metropolis}}
  				\EndFor{}
						\State{$T\gets T\cdot F_c$}
				\EndFor{}
				\EndProcedure{}
  \end{algorithmic}
\end{algorithm}


Der Metropolis-Algorithmus~\ref{alg:metropolis} wird im Rahmen des in Algorithmus~\ref{alg:anneal} dargestellten Annealing-Verfahrens aufgerufen. Es wird mit einer Temperatur von $T_0=1$ begonnen. Dann werden $N_a$ Annealing-Schritte durchgeführt und dabei jeweils $N_c$ Metropolis-Schritte durchgeführt. \\
Diese Prozedur wird für jedes erlaubte $4$-Tupel $\left(a,a_1,b,b_1\right)$ durchgeführt. Diesem Zweck dienen die $4$ Schleifen in Algorithmus~\ref{alg:factorize}. \\
Für jedes Tupel $\left(a,a_1\right)$ wird eine zufällige binäre Zahl $A$ mit $a$ Stellen gezogen, von denen $a_1$ auf $1$ gesetzt sind. Analog wird für jedes Tupel $\left(b,b_1\right)$ eine entsprechende Zahl $B$ erstellt. Anschließend wird die anfängliche Energie berechnet. Außerdem wird das Produkt $A\cdot B$ berechnet, sodass das Programm abgebrochen werden kann, falls zufällig direkt eine Zerlegung der Zahl $N$ in zwei Faktoren gefunden wird. Ansonsten wird das in Algorithmus~\ref{alg:anneal} beschriebene Annealing durchgeführt.

\begin{algorithm}[ht]
  \begin{algorithmic}[1]
				\Procedure{Factorize}{$N$}
    \For{$a\in\left[a_\mathrm{min}, a_\mathrm{max}\right]$}
      \For{$a_1\in\left[1,a\right]$}
						  \State{$A\gets\mathrm{randomBitset}\left(a, a_1\right)$}
        \For{$b\in\left[b_\mathrm{min}, a\right]$}
          \For{$b_1\in\left[1,b\right]$}
  						    \State{$B\gets\mathrm{randomBitset}\left(b, b_1\right)$}
												\State{$E\gets E\left(A,B,E,N\right)$}
												\If{$A\cdot B=N$}
				          \State{Exit}\Comment{Faktoren $A, B$ wurden gefunden}
												\EndIf{}
												\State{$\mathrm{anneal}\left(A,B,N\right)$}
						    \EndFor{}
    				\EndFor{}
						\EndFor{}
				\EndFor{}
				\EndProcedure{}
  \end{algorithmic}
\end{algorithm}


\FloatBarrier{}



\section{Abschätzung der Laufzeit}\label{sec:runtime-theo}
Bei der Untersuchung des Algorithmus ist die Laufzeit in Abhängigkeit von der Länge $n$ der zu faktorisierenden Zahl interessant. Hierzu kann die Anzahl der im ungünstigsten Fall benötigten Metropolis-Schritte betrachtet werden. \\
Schränkt man den Suchbereich nicht wie in Abschnitt~\ref{sec:parameterspace} beschrieben ein, d.h.\ wählt man den Parameterbereich gemäß der Gleichungen~\eqref{eq:parameterspace1}, werden maximal
\begin{equation}
		R_1\left(n\right)=\sum\limits_{a=2}^{n}\sum\limits_{a_1=1}^{a}\sum\limits_{b=2}^{a}\sum\limits_{b_1=1}^{b}N_a\cdot N_c\label{eq:runtime-theo}
\end{equation}
Metropolis-Schritte durchgeführt. Mit dem kleineren Suchbereich gemäß Gleichungen~\eqref{eq:parameterspace2} ergibt sich eine Laufzeit von
\begin{equation}
		R_2\left(n\right)=\sum\limits_{a=a_\mathrm{\min}}^{n}\sum\limits_{a_1=1}^{a}\sum\limits_{b=b_\mathrm{\min}}^{a}\sum\limits_{b_1=1}^{b}N_a\cdot N_c\label{eq:runtime-theo-opt}.
\end{equation}
In beiden Fällen ist grob eine Laufzeit von $\mathcal{O}\left(n^4\cdot N_a\cdot N_c\right)$ zu erwarten, weil die Anzahl der Summanden in jeder Summe linear mit $n$ wächst.

\begin{figure}[!ht]
		\centering
		\includegraphics[height=0.35\textheight]{plot/runtime-theo/runtime.pdf}
		\caption{Theoretische Laufzeiten nach Gln.~\eqref{eq:runtime-theo} und~\eqref{eq:runtime-theo-opt}}\label{fig:runtime-theo}
\end{figure}

Die Gleichungen~\eqref{eq:runtime-theo} und~\eqref{eq:runtime-theo-opt} wurden numerisch ausgewertet und in Abb.~\ref{fig:runtime-theo} graphisch dargestellt.

\begin{figure}[!ht]
		\centering
		\includegraphics[height=0.35\textheight]{plot/runtime-theo/comparison.pdf}
		\caption{Verbesserung der Laufzeit durch Einschränkung des Parameterbereiches}\label{fig:runtime-theo-comparison}
\end{figure}

Außerdem wurde die Abweichung
\begin{equation*}
		1-\frac{R_2\left(n\right)}{R_1\left(n\right)}
\end{equation*}
berechnet und in Abb.~\ref{fig:runtime-theo-comparison} aufgetragen, um die Verbesserung der Laufzeit durch den kleineren Parameterbereich darzustellen. Dabei ist festzustellen, dass durch die Einschränkungen im ungünstigsten Fall $\SI{16}{\percent}$ weniger Metropolisschritte erforderlich sind. \\
Nun vergleichen wir dieses Verhalten mit dem Zahlkörpersieb, welches eine Laufzeit von 
\begin{equation}
		r_\mathrm{sieve}\approx\exp\left(c\cdot{\left(\log N\right)}^\frac{2}{3}{\left(\log\log N\right)}^\frac{1}{3}\right)\label{eq:runtime-sieve}
\end{equation}
mit $c=64/9$ (allgemeines Zahlkörpersieb) hat~\parencite{pomerance}. Diese wurde im Vergleich zu der maximalen Laufzeit der Simulated Annealing Methode in~\ref{fig:runtime-sieve} aufgetragen. Dabei ist zu berücksichtigen, dass das Zahlkörpersieb ein deterministischer Algorithmus ist, also immer das korrekte Ergebnis liefert. Das Simulated Annealing ist spätestens bei der maximalen Laufzeit beendet, muss aber nicht zwangsläufig ein Ergebnis geliefert haben.

\begin{figure}[!ht]
		\centering
		\includegraphics[height=0.35\textheight]{plot/runtime-sieve/plot.pdf}
		\caption{Laufzeitverhalten des allgemeinen Zahlkörpersiebs und der Simulated Annealing Methode}\label{fig:runtime-sieve}
\end{figure}
