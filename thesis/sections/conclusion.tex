\chapter{Fazit und Ausblick}\label{ch:conclusion}
Im Rahmen dieser Arbeit konnte gezeigt werden, dass die von Altschuler und Williams vorgeschlagene Methode~\parencite{altschuler} grundsätzlich geeignet ist, Zahlen in ihre Primfaktoren zu zerlegen. \\
Es konnte das Laufzeitverhalten $\mathcal{O}\left(n^4\cdot N_a \cdot N_c\right)$ mit Hilfe von Simulationen bestätigt werden. Außerdem konnte der Einfluss der Boltzmann-Konstanten und damit der Energieskala auf die Erfolgsrate und die Laufzeit untersucht werden. Mit den Erkenntnissen aus der Simulation ist es möglich, eine automatische Abschätzung von $k_\mathrm{B}$ in Abhängigkeit der Zahlenlänge vorzunehmen. \\
Auf Grund der beschränkten Zeit gibt es noch Untersuchungen, die nicht mehr durchgeführt werden konnten. So könnte untersucht werden, wie sich Erolfgsrate und Laufzeit entwickeln, wenn man zu kleine Zahlen $N_a$ und $N_c$ wählt und somit zunächst nur wenige Programmdurchläufe eine erfolgreiche Zerlegung liefern. \\
Bei der Implementierung des Programmes \textit{factorize} könnte man noch erweiterte Methoden zum Test ob die erhalten Faktoren prim sind einbauen. So existiert zum Beispiel der Test nach Miller und Rabin der zwar eigentlich probabilistisch ist, aber bei geschickter Implementiertung deterministisch ist~\parencite{miller}.
Interessant wäre es auch noch zu untersuchen, welche Primzahlen man preaktischmaximal zerlegen könnte. Dazu könnte man das Programm mit Techniken wie MPI~\parencite{mpi} auf mehrere Knoten eines Clusters zu verteilen, da sich der Code gut parallelisieren lässt.

\vfill
Der Programm-Code der erstellten Programme, die erzeugten Daten und die Arbeit selbst sind zu finden unter:\\
\url{https://github.com/f-koehler/bachelor-thesis}
